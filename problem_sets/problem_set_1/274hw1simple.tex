\documentclass[12pt]{article}
\usepackage{fancyhdr,amsmath,amssymb}
%\usepackage{sabon,hum521}
%\usepackage[euler-digits]{eulervm}
%\usepackage{multicol}
%\usepackage{notsofull}
%\usepackage{graphicx}

\newcommand{\funny}{\ifcase\thepage 
  \or Edward worried about his drinking. Would there be enough gin? Enough ice? -- Donald Barthelme, in \textit{Flying to America}
  \or Pharmaceutical companies are better at inventing diseases that match existing drugs than inventing drugs to match existing diseases. -- Nassim Nicholas Taleb
\fi}

\pagestyle{fancy}
\lhead{}
\rhead{}
\chead{}
\cfoot{\textsf{\tiny\fontseries{m}\selectfont \funny}}
\lfoot{}
\rfoot{}
\renewcommand{\headrulewidth}{0pt}

\DeclareMathOperator{\Span}{Span}
\DeclareMathOperator{\im}{Im}
\DeclareMathOperator{\rk}{rank}
\newcommand{\tto}{\longrightarrow}
\newcommand{\Z}{\ensuremath{\mathbb{Z}}}
\newcommand{\Q}{\ensuremath{\mathbb{Q}}}
\newcommand{\R}{\ensuremath{\mathbb{R}}}
\newcommand{\C}{\ensuremath{\mathbb{C}}}
\newcommand{\F}{\ensuremath{\mathbb{F}}}
\newcommand{\ndiv}{\nmid}
\newcommand{\vct}[2]{\ensuremath{\begin{bmatrix}#1\\#2\end{bmatrix}}}
\newcommand{\vctt}[3]{\ensuremath{\begin{bmatrix}#1\\#2\\#3\end{bmatrix}}}
\newcommand{\vcttt}[4]{\ensuremath{\begin{bmatrix}#1\\#2\\#3\\#4\end{bmatrix}}}
\newcommand{\rvct}[2]{\ensuremath{[#1,#2]}}
\newcommand{\rvctt}[3]{\ensuremath{[#1,#2,#3]}}
\newcommand{\rvcttt}[4]{\ensuremath{[#1,#2,#3,#4]}}
\newcommand{\bE}{\mathbf{e}}
\newcommand{\bv}{\mathbf{v}}
\newcommand{\bw}{\mathbf{w}}
\newcommand{\bu}{\mathbf{u}}
\newcommand{\bx}{\mathbf{x}}
\newcommand{\by}{\mathbf{y}}
\newcommand{\bz}{\mathbf{z}}
\newcommand{\st}{\mathrm{st}}

\newcounter{wowser}
\newcommand{\startwowsers}{\setcounter{wowser}{0}}
\newcommand{\wowser}{\stepcounter{wowser}\par\vspace{\baselineskip}\noindent\textbf{\thewowser.}~}
\newcommand{\swowser}{\stepcounter{wowser}\par\vspace{\baselineskip}\noindent\textbf{*\thewowser.}~}
\renewcommand{\theenumi}{\alph{enumi}}

\newcommand{\look}[1]{\par\vspace{\baselineskip}\noindent\textbf{#1}~}
\newcommand{\sol}{\bgroup\sffamily\par\vspace{\baselineskip}\par}
\newcommand{\esol}{\egroup}

\startwowsers

\begin{document}

\vspace*{\baselineskip}

\centerline{\large\textsc{ma}274, Fall 2021 --- Problem Set 1}

\vspace{2\baselineskip}

These problems are mostly about sets, set membership, and subsets,
reflecting the material in chapter 1 of \textit{Numbers, Space, and
  the Structures of Mathematics}, which I'll denote by NSSM. You
should read chapter 1 carefully, working through the many examples.

Most of these problems ask you to write a proof. The proof structures
provided in NSSM are intended to help you do this. Please use
them. Most of the proofs should be quite short.

This assignment does not need to be done with \LaTeX, but if you are
want to try to do that, go for it. I will place a (simplified) copy of
the \LaTeX\ file for this problem set on the course Moodle page.

Grading for problem sets will work like this:
\begin{enumerate}
\item I will randomly select four problems to be graded.
\item The grader will assign up to two points per question. So if you
  get all questions correct you will have eight points so far.
\item The grader will add up to two ``style points.'' These will
  reflect how well you have explained your answers.
\end{enumerate}
The resulting grade should be interpreted roughly as follows: 9 is a
``perfect score''; 10 means you have done something exceptionally
good; scores above 7 mean you are doing ok; anything less than that is
an alarm bell.

\wowser In NSSM, the notation for an open interval is $(a,b)$. The
notation for a point in $\R^2$ is $(x,y)$, which looks exactly the
same. An alternative is to write $]a,b[$ for the open interval. Is
that notation better or worse? (There is no absolute answer, of
course, but you should state your reasons for preferring one or the
other.)

\wowser Show that the interval $(-1,1)$ is a subset of the interval
$[-2,4]$.

\wowser Show that the interval $(0,1)$ is a subset of the set
$X=\{x\in\R : x^2\leq x\}$. Are the two sets the same?

\wowser NSSM, Section 1.7, Exercise 2.

\wowser NSSM, Section 1.7, Exercise 4.

%\wowser NSSM, Section 1.7, Exercise 6.

\wowser NSSM, Section 1.7, Exercise 7.

\wowser NSSM, Section 1.7, Exercise 11.

%\wowser NSSM, Section 1.7, Exercise 13.

\wowser NSSM, Section 1.7, Exercise 14.

\wowser NSSM, Section 1.7, Exercise 16.

% \wowser NSSM, Section 1.7, Exercise 19.

% \wowser NSSM, Section 1.7, Exercise 21. (The hard part is to show that
% any solution of the differential equation must be $g_A$ for some $A$.)

%\wowser NSSM, Section 1.7, Exercise 22.

\wowser Let $X$ be a set. Explain why these two statements say exactly
the same thing:
\begin{itemize}
\item $a\in X$
\item $\{a\}\subset X$
\end{itemize}
(By ``explain'' I mean that you should write one or two sentences in
English that unpack the meaning of both notations, thereby showing
that they say the same thing.)


\end{document}
